\section{Evaluation}\label{sec:Evaluation}
%Describe the evaluation set up briefly.
% 3 Graph that compares the precision/recall (F1) and time performance of our multiple approaches.
% Our approaches are:
% 1. Graphs for 2D CNN with simple linear projection on one of the axis.
% 2. 2D CNN with an improved transformation from 3D to 2D CNN
% 3. 2D CNN - with - separation and projection and then segmentation with DBSCAN
% 4. 3D CNN
% 5. Different No. of Layers up to 3 and 2 different sizes of the filter.
% - We should run it on a standard machine (better on EC2 because it is better reproducible) and show the processing time performance.
% - We present on our plots, Precision/Recall and Processing time.

We evaluated our implementation \footnote{Github Repository
of our Implementation \url{https://github.com/kiat/debs2019}} using the 4 different experiment
setups:
\begin{enumerate}
  \item 2-Layer CNN on projected data to 2D (Single View) and Object Segmentation with 3D DBSCAN
  \item 2-Layer CNN on projected data to 2D (Using perspective projection) and Object Segmentation with 3D DBSCAN
  \item 4-Layer CNN on projected data to 2D (Single View) and Object Segmentation with 3D DBSCAN
  \item 4-Layer CNN on projected data to 2D (Using perspective projection) and Object Segmentation with 3D DBSCAN
\end{enumerate}

Figure \ref{fig:evaluation2} depicts the precision, recall, accuracy, and processing time of each of the 4 different experiment settings on 
DEBS2019 evaluation system. The Figure \ref{fig:evaluation2} also includes if in testing
all 500 scenes are processed in time or it failed to process them all in the given time limit. The
performance time results compared in Figure \ref{fig:evaluation2} includes end-to-end time include
http  data transmission time from and to the evaluation server. 


%\usepackage{graphics} is needed for \includegraphics
\begin{figure*}[htp]
\begin{center}
  \includegraphics[width=1\linewidth]{images/evaluation2.pdf}
  \caption{Precision, Recall, Accuracy and Processing Time of 4 different our Experiment Variation}
  \label{fig:evaluation2}
\end{center}
\end{figure*}

All of the above settings have a data filtering step. The Figure \ref{fig:accloss} represents accuracy and loss while training the CNN model
on individual objects
using TensorFlow\footnote{\url{https://www.tensorflow.org/} TensorFlow is an open source platform
for machine learning, April 2019} to classify objects into 28 class types.


%\usepackage{graphics} is needed for \includegraphics
% \begin{figure}[htp]
% \begin{center}
%         \includegraphics[scale=0.5]{images/accuracy.png}
%         \caption{Training and Validation Accuracy}
%         \label{fig:accuracy}
% \end{center}
% \end{figure}
% \begin{figure}[htp]
% \begin{center}
%         \includegraphics[scale=0.5]{images/loss.png}
%         \caption{Training and Validation Loss}
%         \label{fig:loss}
% \end{center}
% \end{figure}


\begin{figure}[htp]
\begin{center}
        \includegraphics[scale=1]{images/one_to_one.pdf}
        \caption{Training and Validation Accuracy and Loss}
        \label{fig:accloss}
\end{center}
\end{figure}








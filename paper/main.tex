%
% The first command in your LaTeX source must be the \documentclass command.
\documentclass[sigconf]{acmart}

%
% defining the \BibTeX command - from Oren Patashnik's original BibTeX documentation.
\def\BibTeX{{\rm B\kern-.05em{\sc i\kern-.025em b}\kern-.08emT\kern-.1667em\lower.7ex\hbox{E}\kern-.125emX}}
    
% Rights management information. 
% This information is sent to you when you complete the rights form.
% These commands have SAMPLE values in them; it is your responsibility as an author to replace
% the commands and values with those provided to you when you complete the rights form.
%
% These commands are for a PROCEEDINGS abstract or paper.
\copyrightyear{2018}
\acmYear{2018}
\setcopyright{acmlicensed}
\acmConference[Woodstock '18]{Woodstock '18: ACM Symposium on Neural Gaze Detection}{June 03--05, 2018}{Woodstock, NY}
\acmBooktitle{Woodstock '18: ACM Symposium on Neural Gaze Detection, June 03--05, 2018, Woodstock, NY}
\acmPrice{15.00}
\acmDOI{10.1145/1122445.1122456}
\acmISBN{978-1-4503-9999-9/18/06}

%
% These commands are for a JOURNAL article.
%\setcopyright{acmcopyright}
%\acmJournal{TOG}
%\acmYear{2018}\acmVolume{37}\acmNumber{4}\acmArticle{111}\acmMonth{8}
%\acmDOI{10.1145/1122445.1122456}

%
% Submission ID. 
% Use this when submitting an article to a sponsored event. You'll receive a unique submission ID from the organizers
% of the event, and this ID should be used as the parameter to this command.
%\acmSubmissionID{123-A56-BU3}

%
% The majority of ACM publications use numbered citations and references. If you are preparing content for an event
% sponsored by ACM SIGGRAPH, you must use the "author year" style of citations and references. Uncommenting
% the next command will enable that style.
%\citestyle{acmauthoryear}

%
% end of the preamble, start of the body of the document source.
\begin{document}

%
% The "title" command has an optional parameter, allowing the author to define a "short title" to be used in page headers.
\title{Grand Challenge: A Real-Time Multi-label Classifier for High-Speed Streaming Data}

%
% The "author" command and its associated commands are used to define the authors and their affiliations.
% Of note is the shared affiliation of the first two authors, and the "authornote" and "authornotemark" commands
% used to denote shared contribution to the research.



\author{Sambasiva Rao Gangineni}
\affiliation{%
  \institution{Boston University}  
}
\email{samba693@bu.edu}

\author{Harshad Reddy Nalla}
\affiliation{%
	\institution{Boston University}  
}
\email{harshad@bu.edu}

\author{Dimitrije Jankov}
\affiliation{%
	\institution{Rice University}  
}
\email{dimitrijejankov@gmail.com}

\author{Saeed Fathollahzadeh}
\affiliation{%
	\institution{Iran University of Science and Technology}  
}
\email{fathollahzadeh@comp.iust.ac.ir}

\author{Kia Teymourian}
\affiliation{%
  \institution{Boston University}
 }
\email{kiat@bu.edu}



%
% By default, the full list of authors will be used in the page headers. Often, this list is too long, and will overlap
% other information printed in the page headers. This command allows the author to define a more concise list
% of authors' names for this purpose.
\renewcommand{\shortauthors}{Kia Teymourian, et al.}

%
% The abstract is a short summary of the work to be presented in the article.
\begin{abstract}
The winners of the challenge are announced during the conference. The 2019 DEBS Grand Challenge focuses on the application of machine learning to LiDAR data. The goal of the challenge is to perform classification of objects in different scenes surveyed by the LiDAR. The applications of LIDAR and object detection go well beyond autonomous vehicles and are suitable for use in agriculture, waterway maintenance and flood prevention, and construction.

In this paper, we describe our implementation for ACM DEBS 2019 Grand Challenge for a high-speed online neural network classifier that is highly customized to classify objects from streaming data in real-time.
\end{abstract}

%
% The code below is generated by the tool at http://dl.acm.org/ccs.cfm.
% Please copy and paste the code instead of the example below.
%
\begin{CCSXML}
<ccs2012>
 <concept>
  <concept_id>10010520.10010553.10010562</concept_id>
  <concept_desc>Computer systems organization~Embedded systems</concept_desc>
  <concept_significance>500</concept_significance>
 </concept>
 <concept>
  <concept_id>10010520.10010575.10010755</concept_id>
  <concept_desc>Computer systems organization~Redundancy</concept_desc>
  <concept_significance>300</concept_significance>
 </concept>
 <concept>
  <concept_id>10010520.10010553.10010554</concept_id>
  <concept_desc>Computer systems organization~Robotics</concept_desc>
  <concept_significance>100</concept_significance>
 </concept>
 <concept>
  <concept_id>10003033.10003083.10003095</concept_id>
  <concept_desc>Networks~Network reliability</concept_desc>
  <concept_significance>100</concept_significance>
 </concept>
</ccs2012>
\end{CCSXML}

\ccsdesc[500]{Information Systems~Machine Learning}
\ccsdesc[300]{Information Systems~Data Stream}

%
% Keywords. The author(s) should pick words that accurately describe the work being
% presented. Separate the keywords with commas.
\keywords{neural networks, multi-label classification, data stream processing}



%
% This command processes the author and affiliation and title information and builds
% the first part of the formatted document.
\maketitle

\section{Introduction}
Many real-world applications include multi-lable data and require real-time high-speed multi-lable classification. In most application, it is crucial to proactively react based on classification results. For example, in autonomous car application it is important to detect surrounding cars and pedestrians in real-time to send reaction signals. 

\section{Related Work}\label{sec:relatedWork}
In this brief section, we review some of the most related publications regarding LiDAR point cloud object recognition problem.

%SAEED: describe some related work:
\textbf{3D shape analysis} Performance of 3D shape analysis is heavily dependent on the input representation.
The main representations are volumetric, point cloud and multi-view.

To better compare 3D shape descriptors, we will focus on retrieval performance. Recent
approaches show significant improvements in retrieval. Yavartanoo et al. \cite{DBLP:journals/corr/abs-1811-01571}
introduces multi-view stereographic projection; it first transforms a 3D input volume into a 2D planar image using stereographic projection.

Zhou et al. \cite{Zhou_2018_CVPR} proposed a model that operates only on LiDAR data.
In regard to that, it is the best-ranked model on KITTI \cite{geiger2012we} for 3D and birds-eye view detections using LiDAR data only.
The basic idea is end-to-end learning that operates on grid cells without using handcrafted features. However, even with sparse 3D convolution
operations, this model's computational speed is still slower than other similar proposed architectures.


Wu et al. \cite{DBLP:conf/icra/WuWYK18} present SqueezeSeg which projects point cloud to the front view with cells gridded by LiDAR rotation.
It applies normal 2D CNN for classification and segmentation. The front view representation of point cloud shares
the same multi-scale problem like the camera because the sizes of objects change as the distance varies.

Riegler et al \cite{DBLP:conf/cvpr/RieglerUG17} design more efficient 3D CNN or neural network architectures that exploit sparsity in the point cloud.
However, these CNN based methods still require quantization of point clouds with certain voxel resolution.

Huang et al. \cite{DBLP:conf/icpr/HuangY16} take a point cloud and parse it through a dense voxel grid, generating a set of occupancy voxels which are used as input to a 3D CNN to produce one label per voxel. They map back the labels to the point cloud. Although this approach has been applied successfully, it has disadvantages like quantization, loss of spatial information, and unnecessarily large representations.

Maturana et al. \cite{DBLP:conf/iros/MaturanaS15} used deep learning models is to first convert raw point cloud data into a
volumetric representation, namely a 3D grid. This approach, however, usually introduces quantization artifacts and excessive memory usage, making it difficult 
to capture high-resolution or fine-grained features.


The defined grand challenge 2019 \cite{DEBSGC2019} scenario 
% for an object 
is slightly different from
the above described state-of-the-art because of the following reasons: (i) We need to classify and
count the objects and this is different from semantic segmentation of point clouds. (ii) Objects
can partially cover each other and it is required to classify them with counts of objects.
(iii) Scenes have no relations and are randomly selected.




% KIA: I just put the paper titles here \ldots
% Yavartanoo et al. \cite{DBLP:journals/corr/abs-1811-01571} propose an approach for 3D object classification  named SPNet - it is a deep 3D object classification and retrieval using stereographic projection.

%Wu et al. \cite{DBLP:conf/icra/WuWYK18} propose an approach for 3D LiDAR segmentation. SqueezeSeg: Convolutional Neural Nets with Recurrent {CRF} for Real-Time Road-Object Segmentation from 3D LiDAR Point Cloud.

%Yin et al. \cite{Zhou_2018_CVPR}  VoxelNet: End-to-End Learning for Point Cloud-Based 3D Object Detection.

%Riegler et al \cite{DBLP:conf/cvpr/RieglerUG17} OctNet: Learning Deep 3D Representations at High Resolutions.

%Huang et al. \cite{DBLP:conf/icpr/HuangY16}  Point cloud labeling using a 3D convolutional neural network.


%Maturana et al. \cite{DBLP:conf/iros/MaturanaS15}  VoxNet: {A} 3D convolutional neural network for real-time object recognition.


\section{Data Set}



%SAEED
% The data provided for the challenge consists of point cloud readings simulated for a LiDAR sensor that mounts 64 lasers Fig.\ref{fig:data_overview} a , each shoot 1125 times per rotation. That is, each scene consists of 72,000 readings. Each reading is composed of attributes where X, Y, and Z coordinates are as presented in Fig.\ref{fig:data_overview} b.
% 
% In each scene, objects are representative of urban environments and are of the following types: \textbf{ATM machine}, \textbf{pedestrian}, \textbf{benches}, \textbf{cloth recycling container}, \textbf{drinking fountain}, \textbf{electrical cabinet}, \textbf{emergency phone}, \textbf{fire hydrant}, \textbf{glass recycling container}, \textbf{ice freeze container}, \textbf{mailbox}, \textbf{trash bins}, \textbf{phone booth}, \textbf{trees}, and \textbf{several vehicle types}. In some cases, it is possible for an object in a scene to be hidden from the LiDAR sensor (e.g., when such object is occluded by other objects in the scene)\cite{DEBSGC2019}.



% 
% \begin{figure*}[!ht]
% \begin{center}
%   \includegraphics[width=0.8\textwidth]{./images/GC1.png}
%   \caption{An Overview of Data Set from Point Cloud of a single Scene}
%   \label{fig:data_overview}
% \end{center}
% \end{figure*}

%\usepackage{graphics} is needed for \includegraphics


% \begin{figure}%
% 	\centering
% 	\subfloat[point cloud readings simulated for a LiDAR sensor]{{\includegraphics[width=7cm]{images/data_overview.pdf} }}%
% 	\qquad
% 	\subfloat[a scene with different numbers of objects]{{\includegraphics[width=5cm]{images/GC1.png} }}%
% 	\caption{An Overview of Data Set from Point Cloud of a single Scene}%
% 	\label{fig:data_overview}%
% \end{figure}


%KIA 
%Briefly describe the data set and cite the main grand challange paper. 

%We just need to describe what the data is about. 



\TODO{We need to mention other data sets like The KITTI Dataset \cite{Geiger2013IJRR}}

\section{Architecture}


Describe the data processing pipeline here. 


1. Get the raw data and remove the ground (LiDAR data clean up)

2. Segment data and sent it to classifier 

3. Classification with Neural Network. 



Describe \ldots. 

Different forms of data processing architectures that we have implemented and tested . 

\begin{enumerate}
  \item Different methods for removing noise from raw data (data preprocessing). 
  \item Projecting 3D data into 2D data using 3 different projection methods 
  \item CNN  (with and without maxpool) + fully-connected + dropout + fully-connected+softmax
  \item CNN with different number of hidden layers. 

\end{enumerate}

\section{Evaluation}\label{sec:Evaluation}
%Describe the evaluation set up briefly.
% 3 Graph that compares the precision/recall (F1) and time performance of our multiple approaches.
% Our approaches are:
% 1. Graphs for 2D CNN with simple linear projection on one of the axis.
% 2. 2D CNN with an improved transformation from 3D to 2D CNN
% 3. 2D CNN - with - separation and projection and then segmentation with DBSCAN
% 4. 3D CNN
% 5. Different No. of Layers up to 3 and 2 different sizes of the filter.
% - We should run it on a standard machine (better on EC2 because it is better reproducible) and show the processing time performance.
% - We present on our plots, Precision/Recall and Processing time.

We evaluated our implementation \footnote{Github Repository
of our Implementation \url{https://github.com/kiat/debs2019}} using the 4 different experiment
setups:
\begin{enumerate}
  \item 2-Layer CNN on projected data to 2D (Single View) and Object Segmentation with 3D DBSCAN
  \item 2-Layer CNN on projected data to 2D (Using perspective projection) and Object Segmentation with 3D DBSCAN
  \item 4-Layer CNN on projected data to 2D (Single View) and Object Segmentation with 3D DBSCAN
  \item 4-Layer CNN on projected data to 2D (Using perspective projection) and Object Segmentation with 3D DBSCAN
\end{enumerate}

Figure \ref{fig:evaluation2} depicts the precision, recall, accuracy, and processing time of each of the 4 different experiment settings on 
DEBS2019 evaluation system. The Figure \ref{fig:evaluation2} also includes if in testing
all 500 scenes are processed in time or it failed to process them all in the given time limit. The
performance time results compared in Figure \ref{fig:evaluation2} includes end-to-end time include
http  data transmission time from and to the evaluation server. 


%\usepackage{graphics} is needed for \includegraphics
\begin{figure*}[htp]
\begin{center}
  \includegraphics[width=1\linewidth]{images/evaluation2.pdf}
  \caption{Precision, Recall, Accuracy and Processing Time of 4 different our Experiment Variation}
  \label{fig:evaluation2}
\end{center}
\end{figure*}

All of the above settings have a data filtering step. The Figure \ref{fig:accloss} represents accuracy and loss while training the CNN model
on individual objects
using TensorFlow\footnote{\url{https://www.tensorflow.org/} TensorFlow is an open source platform
for machine learning, April 2019} to classify objects into 28 class types.


%\usepackage{graphics} is needed for \includegraphics
% \begin{figure}[htp]
% \begin{center}
%         \includegraphics[scale=0.5]{images/accuracy.png}
%         \caption{Training and Validation Accuracy}
%         \label{fig:accuracy}
% \end{center}
% \end{figure}
% \begin{figure}[htp]
% \begin{center}
%         \includegraphics[scale=0.5]{images/loss.png}
%         \caption{Training and Validation Loss}
%         \label{fig:loss}
% \end{center}
% \end{figure}


\begin{figure}[htp]
\begin{center}
        \includegraphics[scale=1]{images/one_to_one.pdf}
        \caption{Training and Validation Accuracy and Loss}
        \label{fig:accloss}
\end{center}
\end{figure}









\section{Conclusion and Future Work}\label{sec:conclusion}
We have seen that it is possible to use the LiDAR data to classify objects from the point cloud in real-time.
We have seen that the accuracy of our classification stage is very high when we train and test it on LiDAR point cloud containing only one object.


Clearly, our implementation can be improved and time optimized using all CPUs of a single machine using multi-threading and multi-processing achieve real-time processing time and met application requirements.


Lessons Learned form our implementation are:

\begin{enumerate}
  \item Classification of LiDAR point cloud can achieve high accuracy and real-time processing time by projecting the 3D data into 2D view.
  \item Classification using CNN on point cloud does not need a large number of hidden layers to achieve high accuracy.
  \item CNN may fail to classify if the scene includes tiny objects or objects have variable density like ``Tree Objects''.
  \item If multiple objects are in a scene and they are hiding each other (completely or partially) then object segmentation using DBSCAN or other traditional clustering methods may fail to separate objects.
\end{enumerate}

One of the parts that need improvement, is the segmentation stage. We observed that if the segmentation stage fails to clearly separate objects into single voxels then the subsequent classifier cannot correctly classify the object type. This can be improved by extending our sectioning idea and combination of it with traditional density-based clustering.



\bibliographystyle{ACM-Reference-Format}
\bibliography{refrences}




\end{document}

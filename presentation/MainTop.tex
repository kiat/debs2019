\setbeamersize{text margin left=6mm,text margin right=8mm} 

\usetheme[progressbar=frametitle]{metropolis}

\usepackage{xcolor}

\definecolor{bluegreen}{RGB}{3, 166, 155}
\definecolor{pitchblack}{RGB}{0, 0, 0}
\definecolor{lightbeige}{RGB}{255, 251, 241}
\definecolor{mediumgray}{RGB}{183, 183, 183}
\definecolor{mygreen}{rgb}{0,0.6,0}
\definecolor{mygray}{rgb}{0.5,0.5,0.5}
\definecolor{mymauve}{rgb}{0.58,0,0.82}
\definecolor{keywords}{RGB}{255,0,90}
\definecolor{comments}{RGB}{0,0,113}
\definecolor{red}{RGB}{160,0,0}
\definecolor{green}{RGB}{0,150,0}
\definecolor{navy}{RGB}{0,0,128}



% \setbeamercolor{progress bar}{fg=green,bg=blue}
% \setbeamercolor{background canvas}{bg=pitchblack}
% \setbeamercolor{normal text}{fg=lightbeige}
% \setbeamercolor{frametitle}{bg=bluegreen, fg=mediumgray}
\setbeamercolor{background canvas}{bg=white}
\setbeamercolor{normal text}{fg=black}
\setbeamercolor{frametitle}{bg=white, fg=black}



\usepackage{appendixnumberbeamer}
\usepackage{booktabs}
\usepackage[scale=2]{ccicons}

\usepackage{pgfplots}
\usepgfplotslibrary{dateplot}

\usepackage{xspace}
\newcommand{\themename}{\textbf{\textsc{metropolis}}\xspace}

\usepackage{fancyhdr}
\usepackage[mmddyyyy,hhmmss]{datetime}

% \date{\today}
\date{}
% \date[Last Compiled:]{\today}

\author{Kia Teymourian}
\institute{Boston University \\ \\ \\ \\ \\ \\ \\ \\  Date: \today}
% compiled: \today , Time: \currenttime}



% This is important to set the default font to serif. 
\usefonttheme{serif} % default family is serif

% \setmainfont{Liberation Serif}

% Packages for drawing arrows. 
\usepackage{tikz}
\usepgflibrary{arrows}% for more options on arrows

\usepackage{pgfplots}
\usepgfplotslibrary{dateplot}
\usepackage{xspace}
% \newcommand{\themename}{\textbf{\textsc{metropolis}}\xspace}

\usepackage{blindtext}

\usepackage{listings}

% \usepackage{color}

\usepackage{caption}
\usepackage{graphicx}



\definecolor{backgroundCol}{rgb}{.97, .97, .97}
\definecolor{commentstyleCol}{rgb}{0, 0, 80}
\definecolor{keywordstyleCol}{rgb}{0.737,0.353,0.396}
\definecolor{stringstyleCol}{rgb}{0.192,0.494,0.8}
\definecolor{NumCol}{rgb}{0.686,0.059,0.569}
\definecolor{basicstyleCol}{rgb}{0.345, 0.345, 0.345}       



\lstset{ %
  language=R,                     % the language of the code
  basicstyle=\footnotesize  \ttfamily \color{basicstyleCol},       % the size of the fonts that are used for the code
  % numbers=left,                   % where to put the line-numbers
  % numberstyle=\tiny\color{gray},  % the style that is used for the line-numbers
  stepnumber=1,                   % the step between two line-numbers. If it's 1, each line
                                  % will be numbered
  numbersep=2pt,                  % how far the line-numbers are from the code
  backgroundcolor=\color{white},  % choose the background color. You must add \usepackage{color}
  showspaces=false,               % show spaces adding particular underscores
  showstringspaces=false,         % underline spaces within strings
  showtabs=false,                 % show tabs within strings adding particular underscores
  frame=single,                   % adds a frame around the code
  rulecolor=\color{black},        % if not set, the frame-color may be changed on line-breaks within not-black text (e.g. commens (green here))
  tabsize=1,                      % sets default tabsize to 2 spaces
  captionpos=b,                   % sets the caption-position to bottom
  breaklines=true,                % sets automatic line breaking
  breakatwhitespace=false,        % sets if automatic breaks should only happen at whitespace
  title=\lstname,                 % show the filename of files included with \lstinputlisting;
                                  % also try caption instead of title
  keywordstyle=\color{keywordstyleCol},      % keyword style
  commentstyle=\color{commentstyleCol},   % comment style
  stringstyle=\color{stringstyleCol},      % string literal style
  escapeinside={\%*}{*)},         % if you want to add a comment within your code
  morekeywords={*,...},            % if you want to add more keywords to the set
  belowskip=-0.8 \baselineskip
}


\usepackage{algpseudocode} 
\usepackage{MnSymbol,wasysym}

\definecolor{DarkGrenen}{RGB}{0,100,0}
\definecolor{DarkOliveGreen}{RGB}{85,107,47}

\definecolor{saddlebrown}{RGB}{139,69,19}




\newcommand{\red}[1]{\textcolor{red}{#1}}
\newcommand{\blue}[1]{\textcolor{blue}{#1}}
\newcommand{\green}[1]{\textcolor{DarkGrenen}{#1}}
\newcommand{\brown}[1]{\textcolor{saddlebrown}{#1}}


\newcommand{\redb}[1]{\textcolor{red}{\textbf{\boldmath{#1}}}}
\newcommand{\blueb}[1]{\textcolor{blue}{\textbf{\boldmath{#1}}}}
\newcommand{\greenb}[1]{\textcolor{DarkGrenen}{\textbf{\boldmath{#1}}}}
\newcommand{\brownb}[1]{\textcolor{saddlebrown}{\textbf{\boldmath{#1}}}}

\usepackage{url}


\usepackage{wrapfig}

\usepackage{subcaption}


% \usepackage[parfill]{parskip}





